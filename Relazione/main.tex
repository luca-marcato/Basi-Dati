\documentclass[11pt]{article}

\usepackage{helvet}
\renewcommand{\familydefault}{\sfdefault}

\usepackage{geometry}
\geometry{
   a4paper,
   total={170mm,257mm},
   left=20mm,
   top=20mm,
}

\usepackage{tabularx}

\begin{document}

\begin{center}
    \huge\textbf{Title DB}
\end{center}

\section{\textbf{Abstract}}

\textit{\{nome-sito\}} è un azienda italiana che vuole aprire la propria attività di mercato online
sul territorio nazionale e come tale necessità di un database per la gestione delle sue risorse.\\
Grazie a \textit{\{nome-sito\}} sarà possibile fare acquisti tra una vasta serie di prodotti 
senza doversi recare in negozio, potendo scegliere se ricevere la consegna direttamente a casa
propria oppure specificando durante l'acquisto il punto di ritiro che si preferisce,
per poi andarlo a ritirare personalmente.\\
Ogni utente che desidera fare degli acquisti dovrà registrarsi attraverso la propria email,
aggiungendo i propri dati personali e i metodi di pagamento direttamente nel sito.
Inoltre per chi lo desidera ci si potrà abbonare all'offerta "Prime" per evitare i costi di spedizione
su tutti i prodotti certificati Prime e per poter ricevere le propire consegne in brevissimo tempo
in più verrà fornito l'accesso ad altri servizi esclusivi \textit{\{nome-sito\}}.\\
Visualizzando i propri acquisti sarà possibile conoscere la data di presa in carico della spedizione
e quella di arrivo al punto di consegna. In caso di necessità sarà anche possibile specificare alcune preferenze
di spedizione, in segutio all'avvenuta operazione di acquisto dei propri prodotti.\\
\textit{\{nome-sito\}} assicura ai propri clienti che i fornitori siano accertati mettendo a disposizione
la località di produzione ed un recapito telefonico.

\section{Analisi dei requisiti}

\subsection{Descrizione testuale}

Nella base di dati sono presenti i dati degli \textbf{utenti} registrati, di ogni utente sono noti:
\begin{itemize}
    \item Nome
    \item Cognome
    \item Indirizzo email
    \item Password 
    \item Numero di telefono
    \item Indirizzo di residenza per la consegna
    \item Metodi di pagamento
\end{itemize}
Un utente può abbonarsi all'offerta "Prime", che avrà durata mensile o annuale in base alla scelta dell'utente. L’utente abbonato non paga i costi di
spedizione su tutti i prodotti prime.\\
Le \textbf{carte di credito} utilizzate dall’utente devono essere provviste di:
\begin{itemize}
    \item Numero
    \item Circuito (Mastercard, Visa, …)
    \item Scadenza
    \item CVC
    \item Intestatario
\end{itemize}
I pagamenti vengono autorizzati attraverso un sito esterno, rilevato automaticamente dal circuito.\\
Ogni \textbf{fornitore} deve fornire le informazioni riguardo a:
\begin{itemize}
    \item Nome
    \item Numero di telefono
    \item Indirizzo email
    \item Partita IVA
    \item Indirizzo (Via, Numero civico, CAP, Città, Provincia, Stato) dei propri stabilimenti
\end{itemize} 
Gli stabilimenti dei fornitori possono trovarsi anche in paesi esteri fuori dall'Italia.\\
Di ogni \textbf{prodotto}, inoltre, si vuole conoscere:
\begin{itemize}
    \item Codice
    \item Nome
    \item Prezzo
    \item Quantita' disponibile
    \item Peso
    \item Descrizione
\end{itemize}
Ogni \textbf{ordine} deve memorizzare le informazioni riguardo:
\begin{itemize}
    \item Cliente
    \item Fornitore
    \item Prodotti ordinati 
    \item Metodo di pagamento utilizzato
    \item Data e ora del pagamento
    \item Importo pagato
    \item Indirizzo di consegna
\end{itemize}
Gli ordini possono essere ritirati dal cliente, in quel caso deve essergli fornita la data e l’ora di arrivo della propria consegna al punto di ritiro
scelto dall'utente, oppure possono essere spediti all'indirizzo di residenza dell'utente.\\
Quando viene richiesta una \textbf{spedizione} si vogliono memorizzare le informazioni riguardo:
\begin{itemize}
    \item Data di partenza della spedizione
    \item Data di arrivo dell'ordine
    \item Fascia oraria di arrivo
    \item Orario effettivo di arrivo
\end{itemize}
Infine interessa conoscere i \textbf{punti di consegna}. Per ogni Punto di consegna si vogliono memorizzare le seguenti informazioni:
\begin{itemize}
    \item Indirizzo (Via, Numero civico, CAP, Città, Provincia, Stato)
\end{itemize}
Inoltre, se il punto di consegna e' un punto di ritiro scelto dall'utente (e quindi non il suo indirizzo di residenza), ci interessa sapere:
\begin{itemize}
    \item Orario di apertura
    \item Orario di chiusura
\end{itemize}

\subsection{Glossario dei termini}

\begin{center}
    
    \begin{tabularx}{0.9\textwidth} {
        | >{\raggedright\arraybackslash}X |
          >{\raggedright\arraybackslash}X |
          >{\raggedright\arraybackslash}X |
    }
        \hline
        \textbf{Termine} & \textbf{Descrizione} & \textbf{Collegamenti} \\
        \hline\hline

        %Termine
        Utente &
        %Descrizione
        Utente registrato al sito &
        %Collegamenti
        Carrello, Carta di credito, Residenza \\ 
        \hline

        %Termine
        Prime &
        %Descrizione
        Utente dotato di un abbonamento attivo per poter usufruire delle agevolazioni di spedizione &
        %Collegamenti
        Entità figlia di utente \\ 
        \hline

        %Termine
        Carrello &
        %Descrizione
        Zona di memorizzazione dei prodotti selezionati che non sono ancora stati acquistati  &
        %Collegamenti
        Prodotto, Ordine, Utente\\
        \hline

        %Termine
        Prodotto &
        %Descrizione
        

    \end{tabularx}
\end{center}

\subsection{Operazioni}

\dots

\end{document}