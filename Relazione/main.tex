\documentclass[11pt]{article}

\usepackage{helvet}
\renewcommand{\familydefault}{\sfdefault}

\usepackage{geometry}
\geometry{
   a4paper,
   total={170mm,257mm},
   left=20mm,
   top=20mm,
}

\usepackage{tabularx}
\usepackage{ltablex}

\begin{document}

\begin{center}
    \huge\textbf{Title DB}
\end{center}

\section{\textbf{Abstract}}

\textit{\{nome-sito\}} è un azienda italiana che vuole aprire la propria attività di mercato online
sul territorio nazionale e come tale necessità di un database per la gestione delle sue risorse.\\
Grazie a \textit{\{nome-sito\}} sarà possibile fare acquisti tra una vasta serie di prodotti 
senza doversi recare in negozio, potendo scegliere se ricevere la consegna direttamente a casa
propria oppure specificando durante l'acquisto il punto di ritiro che si preferisce,
per poi andarlo a ritirare personalmente.\\
Ogni utente che desidera fare degli acquisti dovrà registrarsi attraverso la propria email,
aggiungendo i propri dati personali e i metodi di pagamento direttamente nel sito.
Inoltre per chi lo desidera ci si potrà abbonare all'offerta "Prime" per evitare i costi di spedizione
su tutti i prodotti certificati Prime e per poter ricevere le propire consegne in brevissimo tempo.
In più verrà fornito l'accesso ad altri servizi esclusivi \textit{\{nome-sito\}}.\\
Visualizzando i propri acquisti sarà possibile conoscere la data di presa in carico della spedizione
e quella di arrivo al punto di consegna. In caso di necessità sarà anche possibile specificare alcune preferenze
di spedizione, in segutio all'avvenuta operazione di acquisto dei propri prodotti.\\
\textit{\{nome-sito\}} assicura ai propri clienti che i fornitori siano accertati mettendo a disposizione
la località di produzione ed un recapito telefonico.\\
In caso di guasto o per via di una determinata motivazione specificata dall'utente sarà
possibile effettuare il reso dei prodotti acquistati avendo la garanzia di rimborso della spesa. 

\section{Analisi dei requisiti}

\subsection{Descrizione testuale}

Nella base di dati sono presenti i dati degli \textbf{utenti} registrati, di ogni utente sono noti:
\begin{itemize}
    \item Nome
    \item Cognome
    \item Indirizzo email
    \item Password 
    \item Numero di telefono
    \item Indirizzo di residenza per la consegna
    \item Metodi di pagamento
\end{itemize}
Un utente può abbonarsi all'offerta "Prime", che avrà durata mensile o annuale in base alla scelta dell'utente. L'utente abbonato non paga i costi di
spedizione su tutti i prodotti indicati.\\
Le \textbf{carte di credito} utilizzate dall’utente devono essere provviste di:
\begin{itemize}
    \item Numero
    \item Circuito (Mastercard, Visa, …)
    \item Scadenza
    \item CVC
    \item Intestatario
\end{itemize}
I pagamenti vengono autorizzati attraverso un sito esterno, rilevato automaticamente dal circuito.\\
Ogni \textbf{fornitore} deve fornire le informazioni riguardo a:
\begin{itemize}
    \item Nome
    \item Numero di telefono
    \item Indirizzo email
    \item Partita IVA
    \item Indirizzo (Via, Numero civico, CAP, Città, Provincia, Stato) dei propri stabilimenti
\end{itemize} 
Gli stabilimenti dei fornitori possono trovarsi anche in paesi esteri fuori dall'Italia.\\
Di ogni \textbf{prodotto}, si vuole conoscere:
\begin{itemize}
    \item Codice
    \item Nome
    \item Prezzo
    \item Quantità disponibile
    \item Peso
    \item Descrizione
    \item Costo di spedizione
    \item Offerta prime
\end{itemize}
Ogni \textbf{ordine} deve memorizzare le informazioni riguardo:
\begin{itemize}
    \item Carrello (Utente e Prodotti ordinati)
    \item Metodo di pagamento utilizzato
    \item Data e ora di acquisto
    \item Importo pagato
    \item Indirizzo di consegna
    \item Preferenze di spedizione (opzionale)
\end{itemize}
Gli ordini possono essere ritirati dal cliente, in quel caso deve essergli fornita la data e l’ora di arrivo della propria consegna al punto di ritiro
scelto oppure possono essere spediti all'indirizzo di residenza dell'utente.\\
Quando viene richiesta una \textbf{spedizione} si vogliono memorizzare le informazioni riguardo:
\begin{itemize}
    \item Codice
    \item Data di partenza della spedizione
    \item Data di arrivo dell'ordine
    \item Fascia oraria di arrivo
    \item Orario effettivo di arrivo
\end{itemize}
Per ricevere i propri prodotti interessa conoscere i \textbf{punti di consegna}. Per ogni Punto di consegna si vogliono
memorizzare le seguenti informazioni:
\begin{itemize}
    \item Indirizzo (Via, Numero civico, CAP, Città, Provincia, Stato)
\end{itemize}
Inoltre, se il punto di consegna è un punto di ritiro scelto dall'utente (e quindi non il suo indirizzo di residenza), ci interessa sapere:
\begin{itemize}
    \item Orario di apertura
    \item Orario di chiusura
\end{itemize}
Infine per effettuare un \textbf{reso} è necessario specificare:
\begin{itemize}
    \item Utente
    \item Prodotto acquistato
    \item Quantità da rendere
    \item Motivazione 
\end{itemize}

\subsection{Glossario dei termini}

\begin{center}
    
    \begin{tabularx}{0.98\textwidth} {
        | >{\raggedright\arraybackslash}X |
          >{\raggedright\arraybackslash}X |
          >{\raggedright\arraybackslash}X |
    }
        \hline
        \textbf{Termine} & \textbf{Descrizione} & \textbf{Collegamenti} \\
        \hline\hline

        %Termine
        Utente &
        %Descrizione
        Utente registrato al sito &
        %Collegamenti
        Carrello, Carta di credito, Residenza \\ 
        \hline

        %Termine
        Prime &
        %Descrizione
        Utente dotato di un abbonamento attivo che usufruisce delle agevolazioni di acquisto e spedizione &
        %Collegamenti
        Entità figlia di utente \\ 
        \hline

        %Termine
        Carrello &
        %Descrizione
        Zona di memorizzazione dei prodotti selezionati che non sono ancora stati acquistati  &
        %Collegamenti
        Prodotto, Ordine, Utente\\
        \hline

        %Termine
        Prodotto &
        %Descrizione
        Un bene tangibile, venduto nel sito e acquistabile da qualunque utente &
        %Collegamenti
        Fornitore, Carrello, Reso\\
        \hline

        %Termine
        Costo spedizione &
        %Descrizione
        Il costo della spedizione (nel caso di validità "Prime" questo non viene calcolato nell'importo totale) &
        %Collegamenti
        Attributo di Prodotto\\
        \hline

        %Termine
        Importo &
        %Descrizione
        Costo totale relativo ad un ordine. Ogni prodotto comprende anche i costi di spedizione &
        %Collegamenti
        Attributo di Ordine \\ 
        \hline

        %Termine
        Fornitore &
        %Descrizione
        L'ente che produce e mette in vendita i propri prodotti &
        %Collegamenti
        Prodotto, Stabilimento \\ 
        \hline  

        %Termine
        PIVA &
        %Descrizione
        La partita IVA è una sequenza di cifre che identifica univocamente un soggetto che esercita un'attività &
        %Collegamenti
        Attributo di Fornitore \\ 
        \hline

        %Termine
        Stabilimento &
        %Descrizione
        Una sede di produzione di un fornitore &
        %Collegamenti
        Fornitore, Reso \\ 
        \hline
        
        %Termine
        Spedizione &
        %Descrizione
        Tempo che trascorre tra la presa in carico fino al punto di consegna di uno o più ordini &
        %Collegamenti
        Ordine \\ 
        \hline

        %Termine
        Codice &
        %Descrizione
        Codice univoco dal quale è possibile ottenere le date di spedizione del proprio ordine &
        %Collegamenti
        Attributo di Spedizione \\ 
        \hline

        %Termine
        Punto di consegna &
        %Descrizione
        Indirizzo di spedizione &
        %Collegamenti
        Ordine \\ 
        \hline

        %Termine
        Punto di ritiro &
        %Descrizione
        Esercente che fornisce il proprio stabilimento per la ricezione e il ritiro delle consegne a determinati orari &
        %Collegamenti
        Entità figlia di Punto di consegna \\ 
        \hline

        %Termine
        Reso &
        %Descrizione
        La possibilità di ottenere il rimborso su un prodotto acquistato  restituendolo al fornitore &
        %Collegamenti
        Ordine, Prodotto, Stabilimento \\ 
        \hline

    \end{tabularx}
\end{center}

\subsection{Operazioni}

\dots

\end{document}